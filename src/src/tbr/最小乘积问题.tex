\subsubsection{最小乘积问题原理}
\noindent
每个元素有两个权值{x_i}和{y_i}, 要求在某个限制下(例如生成树, 二分图匹配)使得{\Sigma x \Sigma y}最小.对于任意一种符合限制的选取方法, 记$X=\Sigma x_i$, $Y=\Sigma y_i$, 可看做平面内一点$(X,Y)$. 答案必在下凸壳上, 找出该下凸壳所有点, 即可枚举获得最优答案. 可以递归求出此下凸壳所有点, 分别找出距x, y轴最近的两点A, B, 分别对应于$\Sigma y_i$, $\Sigma x_i$最小. 找出距离线段最远的点C, 则C也在下凸壳上, C点满足$AB\times AC$最小, 也即$(X_B-X_A)Y_C + (Y_A-Y_B)X_C - (X_B-X_A)Y_A - (Y_B-Y_A)X_A$最小. 后两项均为常数, 因此将所以权值改成$(X_B-X_A)y_i+(Y_B-Y_A)x_i$,求同样问题(例如最小生成树, 最小权匹配)即可. 求出C点以后, 递归AC, BC.
