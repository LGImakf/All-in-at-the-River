\subsubsection{最大最小定理求纳什均衡点}
\paragraph{最大最小定理} 在二人零和博弈中,
可以用以下方式求出一个纳什均衡点:
在博弈双方中\textbf{任选}一方,
求混合策略 $\mathbf{p}$ 使得对方选择任意一个\textbf{纯策略}时,
己方的最小收益最大(等价于对方的最大收益最小).
据此可以求出双方在此局面下的最优期望得分,
分别等于己方最大的最小收益和对方最小的最大收益. 一般而言, 可以得到形如
$$\max_{\mathbf{p}} \min_i\ \sum_{p_j\in \mathbf{p}}\ p_jw_{i,j}, \mathrm{s.t. } \ p_j\ge 0, \sum p_j=1 $$
的形式. 当 $\sum p_jw_{i,j}$ 可以表示成只与 $i$ 有关的函数 $f(i)$ 时, 可以令初始时 $p_i=0$, 不断调整 $\sum p_jw_{i,j}$ 最小的那个i的概率 $p_i$, 直至无法调整或者 $\sum p_j=1$ 为止.
