\subsubsection{最小割集, 最小割必须边以及可行边}
\noindent

\paragraph{最小割集} 从S出发, 在残余网络中BFS所有权值非0的边(包括反向边),得到点集$\{S\}$, 另一集为$\{V\} - \{S\}$.
\paragraph{最小割集必须点} 残余网络中S直接连向的点必在S的割集中, 直接连向T的点必在T的割集中; 若这些点的并集为全集, 则最小割方案唯一.
\paragraph{最小割可行边} 在残余网络中求强联通分量, 将强联通分量缩点后, 剩余的边即为最小割可行边, 同时这些边也必然满流.
\paragraph{最小割必须边} 在残余网络中求强联通分量, 若S出发可到u, T出发可到v, 等价于$scc_S=scc_u$且$scc_T = scc_v$, 则该边为必须边.

\subsubsection{最大权闭合子图}
\noindent
适用问题: 每个点有点权, 限制条件形如: 选择A则必须选择B, 选择B则必须选择C, D. 建图方式: B向A连边, CD向B连边.\\
求解: S向正权点连边, 负权点向T连边, 其余边容量inf, 求最小割, 答案为S所在最小割集.

\subsubsection{二元关系}
\noinent
适用问题: 有$n$个元素, 每个元素可选A或者B, 各有代价; 有$m$个限制条件, 若元素$i$与$j$的种类不同则产生额外的代价, 求最小代价.\\
求解: S向i连边$A_i$, i向T连边$B_i$, 一组限制$(i,j)$代价为$z$, 则i与j之间连双向容量为$z$的边, 求最小割.

\subsubsection{二分图最小点覆盖和最大独立集}
\noindent
最小点覆盖: 求出一个最大匹配, 从左部开始每次寻找一个未匹配点, 从该点出发可以得到``未匹配-匹配-未匹配..."形式的交错树, 标记所有这些点. 则最小点覆盖方案为右部未标记点与左部标记点的并集. 显然最小点覆盖集合大小 = 最大匹配.\\
最大独立集 = 全集 - 最小点覆盖.

\subsubsection{整数线性规划转费用流}
\noindent
首先将约束关系转化为所有变量下界为0, 上界没有要求, 并满足一些等式, 每个变量在均在等式左边且出现恰好两次, 系数为+1和-1, 优化目标为$maximize\ \ \sum v_ix_i$的形式. 将等式看做点, 等式i右边的值$b_i$若为正, 则S向i连边$(b_i, 0)$, 否则i向T连边$(-b_i, 0)$. 将变量看做边, 记变量$x_i$的上界为$m_i$(无上界则$m_i=inf$), 将$x_i$系数为+1的那个等式u向系数为-1的等式v连边$(m_i, v_i)$.