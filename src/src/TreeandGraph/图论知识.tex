\subsubsection{树链的交}
%	假设两条链$(a_1, b_1), (a_2, b_2)$的lca分别为$c_1, c_2$.
%	再求出$(a_1, a_2), (a_1, b_2), (b_1, a_2), (b_1, b_2)$的lca,
%	记为$d_1, d_2, d_3, d_4$.
%	将$(d_1, d_2, d_3, d_4)$按照深度从小到大排序,
%	$(c_1,c_2)$也从小到大排序.
%	两条链有交当且仅当$dep[c_1] \leq dep[d_1]$且$dep[c_2] \leq dep[d_3]$,
%	则$(d_3, d_4)$是链交的两个端点.
	\inputminted{cpp}{src/yzh/path_inter.cpp}

\subsubsection{动态MST}\
    一个$N$个点$M$条边的无向图, 每次可以修改任意一条边的权值, 在每个修改操作后输出当前最小生成树的边权和. $N,M,Q\leq50000$. 
    我们假设, 有一个图$\{S\}$, 有$k$条边在之后会被修改. 在$k$个MST中, 有些边永远出现在这些MST 中, 而有些边永远不会出现在这些MST中. 我们可以尝试求出这些边, 从而缩小图的规模. 
    \paragraph{找出无用边}将需要修改的边标记为$\infty$, 然后跑MST, 这时不在MST上的且值不为$\infty$边必为无用边, 删除这些边, 减少边数 (注意还原) . 
    \paragraph{找出必须边}将需要修改的边标记为$-\infty$, 然后跑MST, 这时在MST上且不为$-\infty$的边为必须边, 将这些边连接的点合并, 缩小点集 (注意还原) . 
    假设当前区间内需要修改的边数为$k$, 进行删去无用边和找出必须边操作后, 图中最多剩下$k+1$个点和$2k$条边. 如果每次都暴力求MST, 那么时间复杂度为$O(nlg^2n)$; 如果利用排好序的边求MST, 并使用路径压缩+按秩合并的并茶集, 那么时间复杂度为$O(nlgn\alpha (n))$. 

\subsubsection{LCT常见应用}
	\paragraph{动态维护边双}\
		可以通过LCT来解决一类动态边双连通分量问题. 即静态的询问可以用边双连通分量来解决, 而树有加边等操作的问题. 
		\par 把一个边双连通分量缩到LCT的一个点中, 然后在LCT上求出答案. 缩点的方法为加边时判断两点的连通性, 如果已经联通则把两点在目前LCT路径上的点都缩成一个点. 

\subsubsection{差分约束}若要使得所有量两两的值最接近, 则将如果将源点到各点的距离初始化为$0$.  若要使得某一变量与其余变量的差最大, 则将源点到各点的距离初始化为$∞$, 其中之一为$0$.  若求最小方案则跑最长路, 否则跑最短路. 
\begin{comment}
\subsubsection{斯坦纳树}\
		在一个无向带权图$G=(V,E)$中, 将指定的$k$个点连通的一颗树称为\textbf{斯坦纳树}, 边权总和最小的斯坦纳树称为最小斯坦纳树. 
		\par 我们可以用DP+SPFA的方法求解斯坦纳树. 用$F_{i,state}$表示以$i$为根, 指定集合中的点的联通状态为$state$的生成树的最小总权值, 有两种转移方程. 
		\par 第一种, 通过两个子集合并进行转移, 即$F_{i,state}=min(F_{i,subset1} + F_{i,subset2})$, 这一部分使用DP完成. 
		\par 第二种, 在当前的联通状态下, 对该联通状态进行松弛操作, 即${F_{i,state}=min(F_{i,state},F_{j,state}+w(i,j))}$, 这一部分使用SPFA完成. 
        \par 时间复杂度$O(V*3^k+cE*2^k)$, $c$为SPFA复杂度中的常数. 
\end{comment}
\subsubsection{李超线段树}\
        李超线段树可以动态在添加若干条线段或直线$(a_i,b_i)\to (a_j,b_j)$, 每次求$[l,r]$上最上面的那条线段的值. 思想是让线段树中一个节点只对应一条直线, 如果在这个区间加入一条直线, 那么分类讨论. 如果新加的这条直线在左右两端都比原来的更优, 则替换原来的直线, 将原来的直线扔掉. 如果左右两端都比原来的劣, 将这条直线扔掉. 如果一段比原来的优, 一段比原来的劣, 那么判断一下两条线的交点, 判断哪条直线可以完全覆盖一段一半的区间, 把它保留, 另一条直线下传到另一半区间. 时间复杂度$O(nlgn)$. 
\subsubsection{吉如一线段树}\
        吉如一线段树能解决一类区间和某个数取最大或最小, 区间求和的问题. 以区间取最小值为例, 在线段树的每一个节点额外维护区间中的最大值$ma$, 严格次大值$se$以及最大值个数$t$. 现在假设我们要让区间$[L,R]$对$x$取min, 先在线段树中定位若干个节点, 对于每个节点分三种情况讨论: 1, 当$ma\leq x$时, 显然这一次修改不会对这个节点产生影响, 直接退出; 2, 当$se<x<ma$时, 显然这一次修改只会影响到所有最大值, 所以把$num$加上$t*(x-ma)$, 把$ma$更新为$x$, 打上标记退出; 3, 当$se\geq x$时, 无法直接更新着一个节点的信息, 对当前节点的左儿子和右儿子递归处理. 单次操作均摊复杂度$O(lg^2n)$. 

\subsubsection{二分图最大匹配}
        \paragraph{最大独立集 最小覆盖点集 最小路径覆盖} 最大独立集指求一个二分图中最大的一个点集, 使得该点集内的点互不相连. \textbf{最大独立集=总顶点数-最大匹配数}. 最小覆盖点集指用最少的点, 使所有的边至少和一个点有关联. \textbf{最小覆盖点集=最大匹配数}. 最小路径覆盖指一个DAG图$G$中用最少的路径使得所有点都被经过. \textbf{最小路径覆盖=总点数-最大匹配数} (拆点构图). 最大独立集$S$与最小覆盖集$T$互补. 构造方法: 1.做最大匹配, 没有匹配的空闲点$u\in S$ 2.如果$u\in S$那么$u$的邻点必然属于$T$ 3.如果一对匹配的点中有一个属于$T$那么另外一个属于$S$ 4.还不能确定的, 把左子图的放入$S$, 右子图放入$T$.
        \paragraph{二分图最大匹配关键点} 关键点指的是一定在最大匹配中的点. 由于二分图左右两侧是对称的, 我们只考虑找左侧的关键点. 先求任意一个最大匹配, 然后给二分图定向: 匹配边从右到左, 非匹配边从左到右, 从左侧每个不在最大匹配中的点出发DFS, 给到达的那些点打上标记, 最终左侧每个没有标记的匹配点即将爱为关键点. 时间复杂度$O(n+m)$. 
        \paragraph{Hall定理}二分图$G=(X,Y,E)$有完备匹配的充要条件是: 对于$X$的任意一个子集$S$都满足$|S|leq |A(S)|$, $A(S)$是$Y$的子集, 是$S$的邻集. 邻集的定义是与$S$有边的点集. 

\subsubsection{稳定婚姻问题}\
         有$n$位男士和$n$位女士, 每个人都对每个异性有一个不同的喜欢程度, 现在使得每人恰好有一个异性配偶. 如果男士$u$和女士$v$不是配偶但喜欢对方的程度都大于喜欢各自当前的配偶, 则称他们为一个不稳定对. 稳定婚姻问题是为了找出一个不含不稳定对的方案. 
         \par 稳定婚姻问题的经典算法为求婚-拒绝算法, 即男士按自己喜欢程度从高到底依次向每位女士求婚, 直到有一个接受他. 女士遇到比当前配偶更差的男士时拒绝他, 遇到更喜欢的男士时就接受他, 并抛弃以前的配偶. 被抛弃的男士继续按照列表向剩下的女士依次求婚, 直到所有人都有配偶. 算法一定能得到一个匹配, 而且这个匹配一定是稳定的. 时间复杂度$O(n^2)$. 

\subsubsection{最大流和最小割}
	%\paragraph{最小割集求法} 从源点$S$出发, 在残量网络上走, 所有走到的点属于$S$集, 其余点属于$T$集. 
	\paragraph{常见建模方法} 拆点; 黑白染色; 流量正无穷表示冲突; 缩点; 数据结构优化建图; 最小割:每个变元拉一条$S$到$T$的链, 割在哪里表示取值, 相互连边表示依赖关系; 先把收益拿下, 在考虑冲突与代价的影响. 
	%\subparagraph{拆点} 如果一个点有经过次数限制, 可以将它拆成两个点: 一个入点和一个出点. 入点向出点连经过流量为次数限制的边, 然后出点向入点连边即可. 
	%\subparagraph{黑白染色} 一些图有隐含关系, 具有冲突的节点一定能黑白染色出来, 这样可以把黑点放在左部, 白点放在右部来解决问题. 
	%\subparagraph{用流量为正无穷的边表示冲突} 最小割建图中常用, 若左部一点和右部一点只能取一个或者至少取一个, 在它们之间连一条流量为正无穷的边. 
	%\subparagraph{缩点}几个节点的来源或者去向相同, 可以缩为同一个点. 如果从$u$到$v$有一条容量为$\infty$ 的边, 并且点$v$除了点$u$以外没有别的流量来源, 可以把这两个结点合并为一个. 
	%\subparagraph{数据结构优化建图}当一个点同时连一段点的时候, 可以用线段树, ST表等将一段点处理到一起连, 优化建图. 
	%\subparagraph{二者取其一式最小割}每个点都有两种方案可以选择, 每种方案都有一个花费, 并且只能选则其中一种. 另外如果某两个点选择了不同的方案, 还要在这两个点之间增加额外的费用. 新建$S$和$T$表示两种方案, 连边$(S,i,a_i)$或$(i,T,b_i)$表示支持哪一方, 每一对关系连边$(i,j,v_{i,j})$和$(j,i,v_{i,j})$, 求一次最小割即为结果. 
	%\subparagraph{最大权闭合子图/蕴含式最大获利问题}在一个有向图中, 每个点有一个可正可负的点权$a_i$, 对于任意一条有向边$(i,j)$, 如果选了$i$就必须选$j$, 选择一些点使权值最大. 对于点$i$, 如果$a_i$为正, 则连边$(S,i,a_i)$, 否则连边$(i,T,-a_i)$. 对于原图的边$(i,j)$, 连边$(i,j,\infty)$. 则最大权=正权和-最大流. 
	\paragraph{判断一条边是否可能/一定在最小割中}令$G'$为残量网络$G$在强联通分量缩点之后的图. 那么一定在最小割中的边$(u,v)$: $(u,v)$满流, 且在$G'$中$u=S,v=T$; 可能在最小割方案中的边$(u,v)$: $(u,v)$满流, 或$(u,v)$满流, 且在$G'$中$u\neq v$. 
	\paragraph{混合图欧拉回路}把无向边随便定向, 计算每个点的入度和出度, 如果有某个点出入度之差$deg_i=in_i-out_i$为奇数, 肯定不存在欧拉回路. 对于$deg_i>0$的点, 连接边$(i,T,deg_i/2)$;对于$deg_i<0$的点, 连接边$(S,i,-deg_i/2)$. 最后检查是否满流即可. 
	\subsubsection{一些网络流建图}
	\paragraph{无源汇有上下界可行流}
	每条边$(u,v)$ 有一个上界容量$C_{u,v}$和下界容量$B_{u,v}$, 我们让下界变为$0$,上界变为$C_{u,v}-B_{u,v}$, 但这样做流量不守恒. 建立超级源点$SS$和超级汇点$TT$, 用$du_i$来记录每个节点的流量情况, $du_i=\sum B_{j,i}-\sum B_{i,j}$, 添加一些附加弧. 当$du_i>0$时, 连边$(SS,i,du_i)$;当$du_i<0$时, 连边$(i,TT,-du_i)$. 最后对$(SS,TT)$求一次最大流即可, 当所有附加边全部满流时(即$maxflow==所有du_i>0之和$)时有可行解. 
	\paragraph{有源汇有上下界最大可行流}
	建立超级源点$SS$和超级汇点$TT$, 首先判断是否存在可行流, 用无源汇有上下界可行流的方法判断. 增设一条从$T$到$S$没有下界容量为无穷的边, 那么原图就变成了一个无源汇有上下界可行流问题. 同样地建图后, 对$(SS,TT)$进行一次最大流, 判断是否有可行解. 
	如果有可行解, 删除超级源点$SS$和超级汇点$TT$, 并删去$T$ 到$S$的这条边, 再对$(S,T)$进行一次最大流, 此时得到的$maxflow$即为有源汇有上下界最大可行流. 
	\paragraph{有源汇有上下界最小可行流}
	建立超级源点$SS$和超级汇点$TT$, 和无源汇有上下界可行流一样新增一些边, 然后从SS到TT跑最大流. 接着加上边$(T,S,\infty)$, 再从$SS$到$TT$跑一遍最大流. 
	如果所有新增边都是满的, 则存在可行流, 此时$T$到$S$这条边的流量即为最小可行流. 
	\paragraph{有上下界费用流}
	如果求无源汇有上下界最小费用可行流或有源汇有上下界最小费用最大可行流, 用1.6.3.1/1.6.3.2 的构图方法, 给边加上费用即可. 
	求有源汇有上下界最小费用最小可行流, 要先用1.6.3.3的方法建图, 先求出一个保证必要边满流情况下的最小费用. 如果费用全部非负, 那么这时的费用就是答案. 如果费用有负数, 那么流多了可能更好, 继续做从$S$到$T$的流量任意的最小费用流, 加上原来的费用就是答案. 
	\paragraph{费用流消负环}
	新建超级源SS汇TT, 对于所有流量非空的负权边e, 先流满(\texttt{ans+=e.f*e.c, e.rev.f+=e.f, e.f=0}), 再连边SS$\to$e.to, e.from$\to$TT, 流量均为e.f(>0), 费用均为0. 再连边T$\to$S流量$\infty$费用0. 此时没有负环了. 做一遍SS到TT的最小费用最大流, 将费用累加ans, 拆掉T$\to$S的那条边 (此边的流量为残量网络中S$\to$T的流量). 此时负环已消, 再继续跑最小费用最大流.
	\paragraph{二物流}
	水源S1, 水汇T1, 油源S2, 油汇T2, 每根管道流量共用. 求流量和最大.
	建超级源SS1汇TT1, 连边SS1$\to$S1,SS1$\to$S2,T1$\to$TT1,T2$\to$TT1, 设最大流为x1.
	建超级源SS2汇TT2, 连边SS2$\to$S1,SS2$\to$T2,T1$\to$TT2,S2$\to$TT2, 设最大流为x2.
	则最大流中水流量$\frac{x1+x2}{2}$, 油流量$\frac{x1-x2}{2}$.
\begin{comment}
\subsubsection{割点于割边}
        \paragraph{割点 割边} 一个点$u$是割点, 当且仅当: 1. $u$为非树根且有树边$(u,v)$满足$dfn_u\leq low_v$; 2. u为树根且有多于一个的子树. 一条无向边$(u,v)$是桥, 当且仅当$(u,v)$是树边, 且满足$dfn_u<low_v$. 

\subsubsection{2-SAT}
    如果选$A$就必须选$B$就从$A$向$B$连一条边, 如果两个只能选一个的条件在同一个强连通分量中就不合法. 输出可行方案可以比较$X$和$X’$的$bl$的大小, 大的选$X’$. 建图优化一般考虑前后缀的合并. 
\end{comment}
\subsubsection{三元环}\
                有一种简单的写法, 对于每条无向边$(u,v)$, 如果$deg_u<deg_v$, 那么连有向边$(u,v)$, 否则连有向边$(v,u)$(注意度数相等以点标号为第二关键字判断). 然后枚举每个点$x$, 假设$x$ 是三元环中度数最小的点, 然后暴力往后面枚两条边找到点$y$, 判断是否有边$(x,y)$即可. 可以证明, 这样的时间复杂度也是为$O(m\sqrt{m})$的. 
\subsubsection{图同构}
                令$F_t(i)=(F_{t-1}(i)*A+\sum_{i\to j} F_{t-1}(j)*B+\sum_{j\to i} F_{t-1}(j)*C+D*(i-a))mod P$, 枚举点$a$, 迭代$K$次后求得的就是$a$点所对应的$hash$值, 其中$K,A,B,C,D,P$为$hash$参数, 可自选. 

\subsubsection{竞赛图存在 Landau’s Theorem}
    $n$个点竞赛图点按出度按升序排序, 前$i$个点的出度之和不小于$\frac{i(i-1)}{2}$, 度数总和等于$\frac{n(n-1)}{2}$. 否则可以用优先队列构造出方案.

\subsubsection{Ramsey Theorem}
    $6$个人中至少存在$3$人相互认识或者相互不认识. $R(3,3)=6, R(4,4)=18$

\subsubsection{树的计数 Prufer序列}
    树和其prufer编码一一对应, 一颗$n$个点的树, 其prufer编码长度为${n-2}$, 且度数为$d_i$ 的点在prufer 编码中出现${d_i -1}$次. 
    \par 由树得到序列: 总共需要$n-2$步, 第$i$步在当前的树中寻找具有最小标号的叶子节点, 将与其相连的点的标号设为Prufer序列的第$i$个元素$p_i$, 并将此叶子节点从树中删除, 直到最后得到一个长度为$n-2$的Prufer 序列和一个只有两个节点的树. 
    \par 由序列得到树: 先将所有点的度赋初值为$1$, 然后加上它的编号在Prufer序列中出现的次数, 得到每个点的度; 执行$n-2$步, 第$i$步选取具有最小标号的度为$1$的点$u$与$v=p_i$ 相连, 得到树中的一条边, 并将$u$和$v$ 的度减一. 最后再把剩下的两个度为$1$的点连边, 加入到树中. 
    \par 相关结论: $n$个点完全图, 每个点度数依次为$d_1$,$d_2$,...,$d_n$, 这样生成树的棵树为: ${\frac{(n-2)!}{(d_1-1)!(d_2-1)!...(d_n-1)!}}$.\\
    左边有$n_1$个点, 右边有$n_2$个点的完全二分图的生成树棵树为$n_1^{n_2-1}\times n_2^{n_1-1}$. \\
    $m$个连通块, 每个连通块有$c_i$个点, 把他们全部连通的生成树方案数: $(\sum c_i)^{m-2}\prod c_i$
\subsubsection{有根树的计数}\noindent
    首先, 令$S_{n,j}=\sum_{1\leq j\leq n/j}$; 于是$n+1$个结点的有根树的总数为$ a_{n+1}=\frac{\sum_{j=1}^nja_jS_{n-j}}{n}$. 注: $a_1=1,a_2=1,a_3=2,a_4=4,a_5=9,a_6=20,a_9=286,a_11=1842$. 
\subsubsection{无根树的计数}\noindent
    当$n$是奇数时, 有$a_n-\sum_{i}^{n/2}a_ia_{n-i}$种不同的无根树. \\
    当$n$时偶数时, 有$a_n-\sum_{i}^{n/2}a_ia_{n-i}+\frac{1}{2}a_{n/2}(a_{n/2}+1)$种不同的无根树. 
\subsubsection{生成树计数 Kirchhoff's Matrix-Tree Thoerem}
    Kirchhoff Matrix $T=Deg-A$, $Deg$是度数对角阵, $A$是邻接矩阵. 无向图度数矩阵是每个点度数; 有向图度数矩阵是每个点入度.\\
    邻接矩阵$A[u][v]$表示$u$−>$v$边个数, 重边按照边数计算, 自环不计入度数.\\
    无向图生成树计数: $c=|K$的任意1个$n−1$阶主子式$|$\\
    有向图外向树计数: $c=|$去掉根所在的那阶得到的主子式$|$
\subsubsection{有向图欧拉回路计数 BEST Thoerem}
        \[ \mathrm{ec}(G) = t_w(G)\prod_{v \in{V}}(\mathrm{deg}(v) - 1)! \]
        其中$deg$为入度(欧拉图中等于出度), $t_w(G)$为以$w$为根的外向树的个数. 相关计算参考生成树计数.\\
        欧拉连通图中任意两点外向树个数相同: $\mathrm{t_v}(G) = \mathrm{t_w}(G)$.
%\subsubsection{Tutte Matrix}
%    Tutte matrix $A$ of a graph $G=(V,E)$ : 
%    \[A_{ij}=\left\{
%        \begin{aligned}
%            & x_{ij} & \text{if}(i,j)\in E \text{and}i<j
%            & -x_{ij} & \text{if}(i,j)\in E \text{and}i>j
%            & 0 & \text{otherwise}
%        \end{aligned}
%        \right.\]
%    where $x_{ij}$ are indeterminates. The determinant of this skew-symmetric matrix is then a polynomial (in the variables $x_{ij}$, $i<j$ ): this coincides with the square of the pfaffian of the matrix $A$ and is non-zero (as a polynomial) if and only if a perfect matching exists.
%\begin{comment}
    \subsubsection{Edmonds Matrix}
        Edmonds matrix $A$ of a balanced ($|U|=|V|$) bipartite graph $G=(U,V,E)$ : 
        \[A_{ij}=\left\{
            \begin{aligned}
                & x_{ij} & (u_i,v_j)\in E\\
                & 0 & (u_i,v_j)\notin E
            \end{aligned}
            \right.\]
        where the $x_{ij}$ are indeterminates. $G$有完美匹配当且仅当关于$x_{ij}$的多项式$det(A_{ij})$不恒为$0$.
        完美匹配的个数等于多项式中单项式的个数.
%\end{comment}
\subsubsection{Count Acyclic Orientations}
    \noindent
    The chromatic polynomial is a function $P_G(q)$ that counts the number of $q$-colorings of $G$.\\
    Triangle $K_3$ : $t(t-1)(t-2)$\\
    Complete graph $K_n$ : $t(t-1)(t-2)\cdots (t-(n-1))$\\
    Tree with $n$ vertices : $t(t-1)^{n-1}$\\
    Cycle $C_n$ : $(t-1)^{n}+(-1)^{n}(t-1)$\\
    The number of acyclic orientations of an $n$-vertex graph $G$ is $(−1)^nP_G(−1)$, where $P_G(q)$ is the chromatic polynomial of the graph
    $G$.
