如果要求出一个次数为 $k$ 的递推式, 则输入的数列需要至少有 $2k$ 项.

返回的内容满足 $\sum_{j = 0} ^ {m - 1} a_{i - j} c_j = 0$, 并且 $c_0 = 1$ .

如果不加最后的处理的话, 代码返回的结果会变成 $a_i = \sum_{j = 0} ^ {m - 1} c_{j - 1} a_{i - j}$, 有时候这样会方便接着跑递推, 需要的话就删掉最后的处理.

\inputminted{cpp}{src/Math/Berlekamp-Massey.cpp}

如果要求向量序列的递推式, 就把每位乘一个随机权值(或者说是乘一个随机行向量 $v^T$)变成求数列递推式即可. 如果是矩阵序列的话就随机一个行向量 $u^T$ 和列向量 $v$, 然后把矩阵变成 $u^T A v$ 的数列.

\paragraph*{优化矩阵快速幂DP}

	假设$f_i$有$n$维, 先暴力求出$f_{0\textasciitilde 2n - 1}$, 然后跑Berlekamp-Massey, 最后调用快速齐次线性递推即可.

\paragraph*{求矩阵最小多项式}

	矩阵 $A$ 的最小多项式是次数最小的并且 $f(A) = 0$ 的多项式 $f$.

	实际上最小多项式就是$\{A^i\}$的最小递推式, 所以直接调用Berlekamp-Massey就好了, 并且显然它的次数不超过$n$.

	瓶颈在于求出$A^i$, 实际上我们只要处理$A^i v$就行了, 每次对向量做递推.

\paragraph*{求稀疏矩阵的行列式}

	如果能求出特征多项式, 则常数项乘上 $(-1)^n$ 就是行列式, 但是最小多项式不一定就是特征多项式.

	把 $A$ 乘上一个随机对角阵 $B$, 则 $AB$ 的最小多项式有很大概率就是特征多项式, 最后再除掉 $\text{det}\;B$ 就行了.

\paragraph*{求稀疏矩阵的秩}

	设 $A$ 是一个 $n\times m$ 的矩阵, 首先随机一个 $n\times n$ 的对角阵 $P$ 和一个 $m\times m$ 的对角阵 $Q$, 然后计算 $Q A P A^T Q$ 的最小多项式即可.

	实际上不用计算这个矩阵, 因为求最小多项式时要用它乘一个向量, 我们依次把这几个矩阵乘到向量里就行了. 答案就是最小多项式除掉所有$x$因子后剩下的次数.

\paragraph*{解稀疏方程组}

	$Ax = b$, 其中 $A$ 是一个$n \times n$ 的\textbf{满秩}稀疏矩阵, $b$ 和 $x$ 是 $1\times n$ 的\textbf{列}向量, $A, b$ 已知, 需要解出 $x$.

	做法: 显然 $x = A^{-1} b$. 如果我们能求出 $\{A^i b\}$($i \ge 0$)的最小递推式$\{r_{0 \textasciitilde m - 1}\}$($m \le n$), 那么就有结论

	$$ A^{-1} b = -\frac 1 {r_{m - 1}} \sum_{i = 0} ^ {m - 2} A^i b r_{m - 2 - i} $$

	因为$A$是稀疏矩阵, 直接按定义递推出$b \textasciitilde A^{2n - 1} b$即可.
	
	\inputminted{cpp}{src/Math/解稀疏方程组.cpp}