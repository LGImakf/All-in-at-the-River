\paragraph{定义}连接环中不相邻的两个点的边称为弦. 一个无向图称为弦图, 当图中任意长度都大于$3$ 的环都至少有一个弦. 弦图的每一个诱导子图一定是弦图. 
\paragraph{单纯点}一个点称为单纯点当${v}+N(v)$的诱导子图为一个团. 任何一个弦图都至少有一个单纯点, 不是完全图的弦图至少有两个不相邻的单纯点. 
\paragraph{完美消除序列}一个序列${v_1,v_2,...,v_n}$满足$v_i$在${v_i,\cdots,v_n}$的诱导子图中为一个单纯点. 一个无向图是弦图当且仅当它有一个完美消除序列.
\paragraph{最大势算法}最大势算法能判断一个图是否是弦图. 从 $n$ 到 $1$ 的顺序依次给点标号,标号为 $i$ 的点出现在完美消除序列的第 $i$ 个.
设 $\mathrm{label}_i$ 表示第 $i$ 个点与多少个已标号的点相邻, 每次选择 $\mathrm{label}$ 最大的未标号的点进行标号. 
然后判断这个序列是否为完美序列. 如果依次判断 ${v_{i+1},...,.v_n}$ 中所有与 $v_i$ 相邻的点是否构成一个团, 时间复杂度为 $O(nm)$.
考虑优化, 设${v_{i+1},...,v_n}$中所有与$v_i$ 相邻的点依次为$v_{j1}$,...,$v_{jk}$. 
只需判断$v_{j1}$是否与$v_{j2}$,...,$v_{jk}$相邻即可. 时间复杂度$O(n+m)$. 
\paragraph{弦图的染色} 完美消除序列从后往前依次给每个点染色, 给每个点染上可以染的最小的颜色.
\paragraph{弦图的团数} 团数: 最大团的点数. 团数 $\leq$ 色数, 弦图团数 = 色数.  
\paragraph{最大独立集} 完美消除序列从前往后能选就选.
\paragraph{最小团覆盖} 用最少的团覆盖所有的点. 设最大独立集为 $\{p_1,p_2, \dots ,p_t\}$, 
则 $\{p_1\cup N(p_1), \dots , p_t \cup N(p_t)\}$ 为最小团覆盖.
