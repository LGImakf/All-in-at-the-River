\paragraph{定义}我们称连接环中不相邻的两个点的边为弦. 一个无向图称为弦图, 当图中任意长度都大于$3$ 的环都至少有一个弦. 弦图的每一个诱导子图一定是弦图. 
\paragraph{单纯点}设$N(v)$表示与点$v$相邻的点集. 一个点称为单纯点当${v}+N(v)$的诱导子图为一个团. 引理: 任何一个弦图都至少有一个单纯点, 不是完全图的弦图至少有两个不相邻的单纯点. 
\paragraph{完美消除序列}一个序列${v_1,v_2,...,v_n}$满足$v_i$在${v_i,v_{i+1},...,v_n}$的诱导子图中为一个单纯点. 一个无向图是弦图当且仅当它有一个完美消除序列. 
\paragraph{最大势算法}最大势算法能判断一个图是否是弦图. 从$n$到$1$的顺序依次给点标号(标号为$i$ 的点出现在完美消除序列的第$i$个). 设$label_i$表示第$i$个点与多少个已标号的点相邻, 每次选择$label_i$最大的未标号的点进行标号. 
\par 然后判断这个序列是否为完美序列. 如果依次判断${v_{i+1},...,.v_n}$ 中所有与$v_i$相邻的点是否构成一个团, 时间复杂度为$O(nm)$. 考虑优化, 设${v_{i+1},...,v_n}$中所有与$v_i$ 相邻的点依次为$v_{j1}$,...,$v_{jk}$.  只需判断$v_{j1}$是否与$v_{j2}$,...,$v_{jk}$相邻即可. 时间复杂度$O(n+m)$. 
\paragraph{弦图的染色}按照完美消除序列中的点倒着给图中的点贪心染尽可能最小的颜色, 这样一定能用最少的颜色数给图中所有点染色. 弦图的团数=染色数. 
\paragraph{最大独立集}完美消除序列从前往后能选就选. 最大独立集=最小团覆盖. 

\begin{itemize}
\item 团数 $\leq$ 色数 , 弦图团数 = 色数

\item 设 $next(v)$ 表示 $N(v)$ 中最前的点 . 
令 w* 表示所有满足 $A \in B$ 的 w 中最后的一个点 , 
判断 $v \cup N(v)$ 是否为极大团 , 
只需判断是否存在一个 w, 
满足 $Next(w)=v$ 且 $|N(v)| + 1 \leq |N(w)|$ 即可 . 

\item 最小染色 : 完美消除序列从后往前依次给每个点染色 , 
给每个点染上可以染的最小的颜色

\item 最大独立集 : 完美消除序列从前往后能选就选

\item 弦图最大独立集数 $=$ 最小团覆盖数 , 
最小团覆盖 : 
设最大独立集为 $\{p_1,p_2, \dots ,p_t\}$, 
则 $\{p_1\cup N(p_1), \dots , p_t \cup N(p_t)\}$ 
为最小团覆盖
\end{itemize}
